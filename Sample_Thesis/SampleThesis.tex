\documentclass[11pt,english,
%singlespacing,
onehalfspacing,
headsepline]{thesis}

\setlength{\parskip}{0.5\baselineskip}
%\renewcommand{\baselinestretch}{1.5}
%\hoffset=-1.4cm
\usepackage[utf8]{inputenc} % Required for inputting international characters
\usepackage[T1]{fontenc} % Output font encoding for international characters
%\usepackage{lipsum}
\usepackage{blindtext}
\usepackage{palatino}
\usepackage{amsmath,amsfonts,amsthm,amssymb,amsxtra,amscd,color,calligra,mathrsfs}
\usepackage{geometry,graphicx}
\usepackage{a4wide}
\usepackage{enumerate}
%\usepackage{dsfont} % i added for blackboard bold 1's

\usepackage{afterpage} % Use the bibtex backend with the authoryear citation style (which resembles APA)

\usepackage[colorlinks]{hyperref}
\hypersetup{linkcolor=blue,citecolor=blue,filecolor=dullmagenta,urlcolor=blue}
\usepackage[all]{xy}
\usepackage{tikz,tikz-cd,tkz-graph,enumerate}
\usetikzlibrary{matrix,arrows,decorations.pathmorphing}

%%%%%%%%%%%%%%% Customized shorthands %%%%%%%%%%%%%%%
\newcommand{\op}{\operatorname}

\newcommand{\mf}[1]{\mathfrak{#1}}
\newcommand{\mc}[1]{\mathcal{#1}}
\newcommand{\scr}[1]{\mathscr{#1}}
\newcommand{\bb}[1]{\mathbb{#1}}
\newcommand{\dv}{\vee\vee}
\newcommand{\ul}[1]{\underline{#1}}
\newcommand{\dual}{^\vee}

\DeclareMathOperator{\Hom}{\mathcal{H}\!{\it om}}
\DeclareMathOperator{\End}{\mathcal{E}\!{\it nd}}


%%%%%%%%%%% Theorem/Proposition/Lemma/Definition etc. %%%%%%%%%%% 
\numberwithin{equation}{subsection}

\newtheorem{theorem}[equation]{Theorem}
\newtheorem{corollary}[equation]{Corollary}
\newtheorem{lemma}[equation]{Lemma}
\newtheorem{proposition}[equation]{Proposition}

\newtheorem*{thm-nonum}{Theorem}

\theoremstyle{definition}
\newtheorem{definition}[equation]{Definition}
\newtheorem{remark}[equation]{Remark}
\newtheorem{example}[equation]{Example}


%----------------------------------------------------------------------------------------
%	MARGIN SETTINGS
%----------------------------------------------------------------------------------------

\geometry{
	paper=a4paper, % Change to letterpaper for US letter
	inner=2.7cm, % Inner margin=2.5cm
	outer=2.7cm, % Outer margin=2.5cm
	bindingoffset=1.0cm, % Binding offset=1.0cm
	top=1.5cm, % Top margin=2.5cm
	bottom=2.7cm, % Bottom margin=2.5cm
	%showframe, % Uncomment to show how the type block is set on the page
}

% You need to Edit at the appropriate places below: 
\thesistitle{Write Your Thesis Title Here!} % Edit here! 
%\examiner{} 
\degree{Doctor of Philosophy} % Edit here! 
\author{Write Your Name} % Edit here! 
\addresses{} 
\subject{Mathematics} % Edit here!
\keywords{Group; ring; field; differential equation} % Optional 
\supervisor{Your Guide's Name} % Edit Here ...
\university{\href{https://www.iitb.ac.in}{Indian Institute of Technology Bombay}} 
\department{\href{http://www.math.iitb.ac.in}{Department of Mathematics}} 
\faculty{Mathematics Faculty} % Optional

\begin{document}

\frontmatter % Use roman page numbering style (i, ii, iii, iv...) for the pre-content pages

\pagestyle{plain} 

\begin{titlepage}
\begin{center}

\vspace*{.03\textheight}
\includegraphics[width=50mm]{IITB-logo.jpg} % Insert Institute logo. 

\HRule \\[0.4cm] % Horizontal line
{\huge \bfseries \ttitle\par}\vspace{0.4cm} % This automatically includes thesis title here! 
\HRule \\[1.5cm] % Horizontal line

\large \textit{A thesis submitted to the} \\ 
Indian Institute of Technology Bombay, Mumbai\\
\textit{for the degree of \degreename \\ in Mathematics \\ by}\\ [1.0cm] 

\Large \authorname \\ 
\large \deptname \\ \large \univname \\ Mumbai 400 076, India

 
\vfill

{\large Month, Year}\\[1cm] % Write Date/Month/Year here. 
%{\large \today}\\[1cm] % To use today's date automatically. 

\end{center}

\end{titlepage}

%----------------------------------------------------------------------------------------
%	Decorations
%----------------------------------------------------------------------------------------
%\cleardoublepage 
% \vspace*{75px}

%\begin{center}
%\includegraphics[width=170mm]{declaration.jpg}
%\end{center}

\begin{declaration}

\noindent
This thesis is a presentation of my original research work. Wherever contributions of others 
are involved, every efforts are made to indicate this clearly, with due reference to the 
literature, and acknowledgement of collaborative research and discussions.  \\ 

\noindent 
This work was done under the guidence of Professor [Guide's name], at the Department of Mathematics 
of the Indian Institute of Technology Bombay, Mumbai, India. I have collaborated with [Collaborator 1], 
[Collaborator 2], and [Collaborator 3] while working towards this thesis. \\ \\ \\  


\hfill{FirstName LastName \hspace*{60px}} 

\hfill{\textbf{[Candidate's name and signature]}} \\ \\

\noindent
In my capacity as supervisor of the candidate's thesis, I certify that the above statements 
are true to the best of my knowledge. \\ \\ \\ \\ \\


%\hspace*{20px}  
\noindent
\hspace*{10px}\supname % This will insert Supervisor's name here automatically! 

\noindent
\textbf{[Guide's name and signature]} \\ 

\noindent 
Date: 

\end{declaration}
%----------------------------------------------------------------------------------------
%	Abstract
%----------------------------------------------------------------------------------------

%\begin{abstract}
%\addchaptertocentry{\abstractname} 
%In this thesis, we study..... 
%\end{abstract}

%----------------------------------------------------------------------------------------
%	Acknowledgements
%----------------------------------------------------------------------------------------

\begin{acknowledgements}
%\addchaptertocentry{\acknowledgementname} 
%{\small 
Write acknowledgements here...!!! 
%}

\end{acknowledgements}

% DEDICATION

\dedicatory{To someone \ldots \\ 
\hspace{2cm} you want to dedicate the thesis ... } 

%%%%%%%%%%%%%%%%%%%%%%%%%%%%%%
% Creating table of contents %
%%%%%%%%%%%%%%%%%%%%%%%%%%%%%%
\cleardoublepage % This is required to correctly locate the table of contents from the index of output pdf file. 
\pdfbookmark{\contentsname}{Contents} % Creates index of table of contents in the output pdf file. 
\tableofcontents % This generates the table of contents automatically! 

%----------------------------------------------------------------------------------------
%	THESIS CONTENT - CHAPTERS
%----------------------------------------------------------------------------------------

\mainmatter % Begin numeric (1,2,3...) page numbering

\pagestyle{thesis}

%%%%%%%%%%%%%%%%%%%%%%%%%%%%%%%%%%%% CHAPTER 1 %%%%%%%%%%%%%%%%%%%%%%%%%%%%%%%%%%%%%%%%%%
\setcounter{chapter}{0}
\chapter*{\centering Introduction} % This will create a Chapter without any number! 
% ``\centering" is used to place "Introduction" at the center of the line. 
\addcontentsline{toc}{chapter}{Introduction} % This is to add this chapter in the table of contents. 
Example of a Chapter without any number! 

% End of Chapter "Introduction". 

%%%%% Chapter 1: Preliminaries %%%%%%%
\begin{chapter}{Preliminaries}\label{ch-1}
This is a \LaTeX\ template for a Ph.D. thesis. You can modify it as you wish. 

\section{First Section's Name}\label{sec-1.1}
\subsection{Subsection Name}
Sample \TeX. 

\end{chapter} % End of Chapter 1. 


% Beginning of Chapter 2. 
\begin{chapter}{Name of chapter 2}\label{ch-2}
In this chapter, we study .... 
\section{Name of section 2.1}\label{sec-2.1}
In this section, we study .... 

\subsection{Name of subsection}\label{sec-2.1.1}
Write something here...!!! 

\section*{Section without number}\label{sec-2.2-nonum}
This is an example of section without number! 

\end{chapter}

\cleardoublepage\pdfbookmark[-1]{\bibname}{bibliography}
\phantomsection\addcontentsline{toc}{chapter}{\bibname}
\begin{thebibliography}{AAAA}

\bibitem[Ram96]{Ra2} A. Ramanathan, Moduli for principal bundles over algebraic curves. II, 
\textit{Proc. Indian Acad. Sci. Math. Sci.} {\bf 106} (1996), 421--449. 

\bibitem[Spr98]{Sp} T. A. Springer, \textit{Linear algebraic groups}, Second edition, 
Progress in Mathematics, \textbf{9}, \textit{Birkhäuser Boston, Inc., Boston, MA}, (1998). 

\bibitem[Sim92]{Si} C. T. Simpson, Higgs bundles and local systems, 
\href{http://www.numdam.org/item?id=PMIHES_1992__75__5_0}{\textit{Inst. Hautes \'{E}tudes 
Sci. Publ. Math.} \textbf{75} (1992), 5--95.} 

\bibitem[Sor00]{So} C. Sorger, Lectures on moduli of principal $G$--bundles over algebraic curves, 
\href{http://users.ictp.it/\~pub_off/lectures/lns001/Sorger/Sorger.pdf}{\textit{School on Algebraic 
Geometry (Trieste, 1999)}, 1--57, ICTP Lect. Notes, 1, \textit{Abdus Salam Int. Cent. Theoret. Phys., 
Trieste}, (2000)}. 

\end{thebibliography}

\end{document}
