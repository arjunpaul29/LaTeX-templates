%%%%%%%%%%%%%%%%%%%%%%%%%%%%%%%%%%%%%%%%%%%%%%
\documentclass[11pt,leqno]{amsart}
\setlength{\textheight}{23cm}
\setlength{\textwidth}{15.5cm}
\setlength{\topmargin}{-0.8cm}
\setlength{\parskip}{0.3\baselineskip}
\hoffset=-1.4cm

\usepackage{palatino} % For palatino style, may be used/removed. 
\usepackage{amsmath,amssymb,amsxtra,color,calligra,mathrsfs,comment,url}
%\let\circledS\undefined % here - PS
%\usepackage[bitstream-charter]{mathdesign}

\usepackage[tableposition=above]{caption}

\usepackage{xcolor} % Required for specifying custom colours
\colorlet{mdtRed}{red!50!black}
\colorlet{dblue}{blue!50!black}
\usepackage[colorlinks,pagebackref=true]{hyperref}
\hypersetup{linkcolor=dblue,citecolor=dblue,filecolor=dullmagenta,urlcolor=mdtRed}

\renewcommand*{\backref}[1]{}
\renewcommand*{\backrefalt}[4]{[{%
		\ifcase #1 Not cited.%
		\or $\uparrow$~#2.%
		\else $\uparrow$~#2.%
		\fi%
	}]}

\usepackage[all]{xy}
\usepackage{tikz,tikz-cd,tkz-graph,enumerate}
\usetikzlibrary{matrix,arrows,decorations.pathmorphing}

%%%%%%%%%%%%%%% Some shortcuts %%%%%%%%%%%%%%%%%
\DeclareMathOperator{\Hom}{\textnormal{Hom}}
\DeclareMathOperator{\sHom}{\mathcal{H}\!\textit{om}}
\DeclareMathOperator{\sEnd}{\mathcal{E}\!\textit{nd}}

\newcommand{\mf}[1]{\mathfrak{#1}}
\newcommand{\mc}[1]{\mathcal{#1}}
\newcommand{\scr}[1]{\mathscr{#1}}
\newcommand{\bb}[1]{\mathbb{#1}}
\newcommand{\dv}{\vee\vee}

\newcommand{\xrrarrow}[3][2]{\overset{#2}{\underset{#3}{\rightrightarrows}}}
\newcommand{\xllarrow}[3][2]{\overset{#2}{\underset{#3}{\leftleftarrows}}}

\numberwithin{equation}{subsection}

\newtheorem{theorem}[equation]{Theorem}
\newtheorem{corollary}[equation]{Corollary}
\newtheorem{lemma}[equation]{Lemma}
\newtheorem{proposition}[equation]{Proposition}
\newtheorem*{theorem-nonumber}{Theorem}

\theoremstyle{definition}
\newtheorem{definition}[equation]{Definition}
\newtheorem{remark}[equation]{Remark}
\newtheorem{example}[equation]{Example}

\newtheorem*{theorem-nonum}{Theorem}%[section]
\newtheorem*{corollary-nonum}{Corollary}

%%%%%% Making Title in normal font 
\newenvironment{nouppercase}{%
	\let\uppercase\relax%
	\renewcommand{\uppercasenonmath}[1]{}}{}

\newcommand{\titlefont}{\fontsize{28}{22}\selectfont} % To customize font size for title. 
\makeatletter
\def\@settitle{
	\begin{center}%
		\baselineskip14\p@\relax
		%\bfseries
		\normalfont\titlefont
		\@title
	\end{center}%
}
\makeatother

%%%%%%%% Symbols/letters/numbers for \thanks{} command: 
\makeatletter
\newcommand\thankssymb[1]{\textsuperscript{\@fnsymbol{#1}}}
\newcommand\thanksletter[1]{\lowercase{\textsuperscript{\@alph{#1}}}}
\newcommand\thanksnum[1]{\textsuperscript{#1}}
\makeatother

%%%%%%%% Address, Email, Corresponding author: 
\renewcommand{\email}[2][1]{\thanks{\textit{Email address}#1: \href{mailto:#2}{#2}}}
\renewcommand{\address}[2][1]{\thanks{\textit{Address}#1: #2}} 
%\newcommand{\corauth}[2][1]{\thanks{#2Corresponding author: #1}} 
\newcommand{\corauth}[2][]{\thanks{#2Corresponding author}} 

\begin{document}

\baselineskip=15.5pt

\title[Basics of \LaTeX]{An introduction to \LaTeX} 

\author[F. Author]{\large First Author\thanksnum{1}\thankssymb{1}}
\address[\thanksnum{1}]{Department of Mathematics, Indian Institute of Technology Bombay, Powai, Mumbai 400076, India}
\email[\thanksnum{1}]{author1@iitb.ac.in}

\author[S. Author]{Second Author\thanksnum{2}} 
\address[\thanksnum{2}]{Department of Mathematics, Indian Institute of Technology Bombay, Powai, Mumbai 400076, India}
\email[\thanksnum{2}]{author2@iitb.ac.in}

\author[T. Author]{Third Author\thanksnum{3}} 
\address[\thanksnum{3}]{Department of Mathematics, Indian Institute of Technology Bombay, Powai, Mumbai 400076, India}
\email[\thanksnum{3}]{author3@iitb.ac.in}

\corauth[]{\thankssymb{1}} 

\subjclass[2010]{14J60, 53C07, 32L10} 

\keywords{\LaTeX} 

\date{\today}

%\maketitle

\begin{nouppercase}
	\maketitle
\end{nouppercase}

\tableofcontents

\begin{abstract}
In this article, we learn some basics of writing documents using \LaTeX. 
\end{abstract}


\section{Advice} 
To become comfortable in writing documents using \LaTeX\, you must practice it as much as you can. 
Whenever you face some problem in writing particular type setting / style / diagram / table etc., 
use Google search to find out what you are looking for. There are many well written webpages, 
where you can find solutions for most of the problems you will be facing at the beginning. 
This is a continuous process, and may takes years of time to become comfortable with using \LaTeX. 
So don't give up at the beginning, and keep practising. 


\section{Basics of a \LaTeX\ document}
\subsection{Different types of fonts}
Example of various font styles: 
{\it italics}, {\bf bold}, {\tt type writer font} etc. \\ 
In math mode, you can use: $R, \mathbf{R}, \mathbb{R}, \mathcal{R}, \mathscr{R}, \mathfrak{R}$ etc. \\ 
Tiny font: {\tiny a b c d..} \\ 
Small font: {\small a b c d..} \\ 
Large font: {\large a, b, c, d..} \\ 
Extra large font: {\Large a, b, c, d...} \\ 
Huge font: {\huge a, b, c, d...} \\ 
Extra huge font: {\Huge a, b, c, d...} \\ 
For more details, see \href{https://en.wikibooks.org/wiki/LaTeX/Fonts}{https://en.wikibooks.org/wiki/LaTeX/Fonts}. \\ 
Accents: \'a, \`a, \"a etc. \\ 
Subscript and superscript: $a^2, x_1, x_2$ etc. \\ 
Colouring text: {\color{red} red}, {\color{mdtRed} dark-red}, {\color{blue} blue}, {\color{dblue} dark-blue} etc. \\ 


\subsection{Equations}
Example of an equation: 
\begin{equation}
(x+y+z)^2 = x^2+y^2+z^2+2xy+2yz+2zx\,. 
\end{equation}

\noindent
When you want to write an equation, which is not fitting into a line, you should use: 
\begin{align}
(x_1 + x_2 + x_3 + x_4 + x_5 + x_6 + x_7)^2 
& = x_1^2 + x_2^2 + x_3^2 + x_4^2 + x_5^2 + x_6^2 \nonumber \\ 
& + 2(x_1x_2 + x_2x_3 + x_3x_4 + x_4x_5 + x_5x_6) \nonumber \\ 
& + 2(x_1x_3 + x_2x_4 + x_3x_5 + x_4x_6)  \\ 
& + 2(x_1x_4 + x_2x_5 + x_3x_6) + 2(x_1x_5 + x_2x_6) + 2x_1x_6 \nonumber 
\end{align}

\noindent
Example of equation array: 

\begin{eqnarray}
A + B + C & = & C + B + A \nonumber \\ 
& = & B + A + C  \\ 
& = & A + C + B \nonumber
\end{eqnarray}

\noindent
Example of a matrix: 
\begin{equation}
A = \begin{pmatrix}
a_{11} & a_{12} & a_{13} \\ 
a_{21} & a_{22} & a_{23} \\ 
a_{31} & a_{32} & a_{33} 
\end{pmatrix} 
\end{equation}

\noindent
Example of two matrices side by side: 
\begin{equation}
A = \begin{pmatrix}
a_{11} & a_{12} & a_{13} \\ 
a_{21} & a_{22} & a_{23} \\ 
a_{31} & a_{32} & a_{33} 
\end{pmatrix} 
\ \ \ \text{and}\ \ \ 
B = \begin{bmatrix}
a_{11} & a_{12} & a_{13} \\ 
a_{21} & a_{22} & a_{23} \\ 
a_{31} & a_{32} & a_{33} 
\end{bmatrix}
\end{equation}

\noindent
Example of a multiline equation: 
\begin{equation}
f(x) = \left\{\begin{array}{rcl}
x^2 & \mbox{if} & 0 \leq x < \infty, \\ 
x^3 & \mbox{if} & x \leq 0. 
\end{array}\right. 
\end{equation}


\subsection{Theorems, Proposition, Lemma, Corollary etc.}

\begin{proposition}\label{prop-1}
	Let $A$ be a commutative ring with identity. Let $\mathfrak a$ be an ideal of $A$. 
	\begin{enumerate}[(a)]
		\item $\mathfrak a$ is a prime ideal of $A$ if and only if $A/\mathfrak a$ is an integral domain. 
		\item $\mathfrak a$ is a maximal ideal of $A$ if and only if $A/\mathfrak a$ is a field. 
	\end{enumerate}
\end{proposition}


\begin{theorem}
	Let $A$ be a commutative ring with identity. Let $\mathfrak a$ be an ideal of $A$. 
	\begin{enumerate}[(i)]
		\item $\mathfrak a$ is a prime ideal of $A$ if and only if $A/\mathfrak a$ is an integral domain. 
		\item $\mathfrak a$ is a maximal ideal of $A$ if and only if $A/\mathfrak a$ is a field. 
	\end{enumerate}
\end{theorem}

\begin{proof}
	\begin{enumerate}[(i)]
		\item Write a proof here. 
		
		\item Write a proof here. 
	\end{enumerate}
\end{proof}

\begin{lemma}
	Let $A$ be a commutative ring with identity. Let $\mathfrak a$ be an ideal of $A$. 
	\begin{enumerate}[(I)]
		\item $\mathfrak a$ is a prime ideal of $A$ if and only if $A/\mathfrak a$ is an integral domain. 
		\item $\mathfrak a$ is a maximal ideal of $A$ if and only if $A/\mathfrak a$ is a field. 
	\end{enumerate}
\end{lemma}

Now we write a proof of Proposition \ref{prop-1} here. 
\begin{proof}[Proof of Proposition \ref{prop-1}]
	Write a proof of Proposition \ref{prop-1} here. 
\end{proof}

\begin{corollary}
	content...
\end{corollary}

\begin{remark}
	content...
\end{remark}


\subsection{Writing diagrams}
Here is an example of various diagrams. 

\begin{enumerate}[(i)]
	\item A simple diagram using \texttt{tikzcd}:  
	\begin{equation}
	\begin{tikzcd}
	A \ar{r}{f} \ar{d}{g} & B \ar{d}{h} \\ 
	C \ar{r}{i} & D
	\end{tikzcd}
	\end{equation}
	
	\item Drawing diagram with curved arrows usign \texttt{tikzcd}: 
	\begin{equation}
	\begin{gathered}
	\begin{tikzcd}
	Z \arrow[bend right=15]{rdd}[below=8]{a} \arrow[bend left=15]{rrrd}{b} \ar[dashed]{rd}{c} \\
	& X \times_k Y \ar{rr}[below]{f_Y} \ar{d}{g_X} && Y \ar{d}[left]{g} \\
	& X \ar{rr}[below]{f} && k
	\end{tikzcd}
	\end{gathered}
	\end{equation}
	
	\item Drawing diagram with curved arrows usign \texttt{xymatrix}: 
	\begin{equation}
	\begin{gathered}
	\xymatrix{
		Z \ar@/_1pc/[ddr]_f \ar@{-->}[dr]^\phi \ar@/^1pc/[drrr]^g & & & \\ 
		& X \times_k Y \ar[rr]^{p_2} \ar[d]^{p_1} & & Y \ar[d]^\xi \\ 
		& X \ar[rr]^\eta & & k
	}
	\end{gathered}
	\end{equation}
	
	
	\item Diagram using \texttt{xymatrix} package. 
	\begin{equation}
	\begin{gathered}
	\xymatrix{
		0 \ar[r] & A_1 \ar@{^(->}[r]^{f_1} \ar@{=}[d]_{v_1} & A_1\oplus A_2 \ar@{->>}[r]^{f_2}\ar[d]_{v_2}^{\simeq} 
		& A_2 \ar[r] \ar[d]_{v_3}^{v_3'} & 0 \\ 
		0 \ar@{-->}[r] & B_1 \ar@{..>}[r]^{g_1} & B_2 \ar@{~>}[r]^{g_2} & B_3 \ar[r] & 0 
	}
	\end{gathered}
	\end{equation}
	
	\item Drawing Graph using \texttt{tikzpicture}: 
	\begin{center}
		\begin{tikzpicture}
		[scale=0.8,auto=left,every node/.style={circle,fill=blue!100}]
		\node (n6) at (1,10) {6};
		\node (n4) at (4,8)  {4};
		\node (n5) at (8,9)  {5};
		\node (n1) at (11,8) {1};
		\node (n2) at (9,6)  {2};
		\node (n3) at (5,5)  {3};
		
		\foreach \from/\to in {n6/n4,n4/n5,n5/n1,n1/n2,n2/n5,n2/n3,n3/n4}
		\draw (\from) -- (\to); 
		\end{tikzpicture}
	\end{center}
	
\end{enumerate}



\subsection{Creating tables}
Here we show some simple examples of writing table in \LaTeX\. For more complicated tables, according to your requirements, 
you can see the following page: \\
\noindent 
\href{https://www.overleaf.com/learn/latex/Tables}{https://www.overleaf.com/learn/latex/Tables}. 

\begin{table}[h!]
	\centering
	\begin{tabular}{|c|c|c|c|}
		\hline % draw a horizontal line. 
		1 & A & B & C \\ 
		\hline
		2 & D & E & F \\ 
		\hline
		3 & G & H & I \\ 
		\hline
	\end{tabular}
	\caption{Simple Table}
	\label{table:0}
\end{table}

\noindent
Here is an example of two tables placed side by side. 
\begin{center}
	\begin{table}[h!] % To place the table at this precise location, use the parameter h!. 
		\parbox{.45\linewidth}{
			\centering 
			\begin{tabular}{|c|c|c|c|}
				\hline
				\multicolumn{4}{|c|}{Sample space 1} \\ 
				\hline
				1 & A & B & C \\ 
				\hline
				2 & D & E & F \\ 
				\hline
				3 & G & H & I \\ 
				\hline
			\end{tabular}
		\caption{1st List of items}
		\label{table:1} 
		}
		\hfill
		\parbox{.45\linewidth}{
			\centering 
			\begin{tabular}{|c|c|c|c|}
				\hline
				\multicolumn{4}{|c|}{Sample space 2} \\ 
				\hline
				1 & A & B & C \\ 
				\hline
				2 & D & E & F \\ 
				\hline
				3 & G & H & I \\ 
				\hline
			\end{tabular}
			\caption{2nd List of items}
			\label{table:2}
		}
	\end{table}
\end{center}


%\newpage 
\section{Acknowledgements}
\small{The authors would like to thank ``Name Surname'' for useful discussions. 
The first named author is supported by ``name of funding agency''. } 



\begin{thebibliography}{AA}
	\providecommand{\doi}[2][]{doi: \href{https://doi.org/#2}{#2}}
	\providecommand{\arxiv}[2][]{arXiv:\href{https://arxiv.org/abs/#2}{#2}} 
	
	\bibitem{Atiyah-1957}
	M. F. Atiyah, Complex analytic connections in fibre bundles, 
	\textit{Trans. Amer. Math. Soc.}, \textbf{85} (1957), 181--207. 
	\doi{10.2307/1992969}. 
	
	\bibitem{Deligne-1970}
	Pierre Deligne, \textit{\'Equations diff\'erentielles \`a points singuliers r\'eguliers}, 
	Lecture Notes in Mathematics, Vol. 163, Springer-Verlag, Berlin-New York (1970). 
	\doi{10.1007/BFb0061194}. 
	
	\bibitem{Hartshorne-1977}
	Robin Hartshorne, \textit{Algebraic geometry}, Graduate Texts in Mathematics, No. 52. 
	\textit{Springer-Verlag, New York-Heidelberg}, 1977.  
	\doi{10.1007/978-1-4757-3849-0}. 
	
	\bibitem{Voisin-I}
	Claire Voisin, \textit{Hodge theory and complex algebraic geometry. I}, 
	\textit{Cambridge Studies in Advanced Mathematics}, volume~76, 
	Cambridge University Press, Cambridge, english edition (2007). 
	\doi{10.1017/CBO9780511615344}. 
	
	\bibitem{Weil-1938}
	Andr\'e Weil, Généralisation des fonctions abéliennes, 
	\textit{J. Math. Pures Appl.}, \textbf{17} (1938), 47--87. 
	
\end{thebibliography}

\end{document}
